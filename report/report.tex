\documentclass{article}
\usepackage[english,ukrainian]{babel}
\usepackage[letterpaper,top=2cm,bottom=2cm,left=3cm,right=3cm,marginparwidth=1.75cm]{geometry}
\usepackage{amsmath}
\usepackage{graphicx}
\usepackage[colorlinks=true, allcolors=blue]{hyperref}

\title{Реалізація структури графу}
\author{ФІ-13 Дідух Максим}

\begin{document}
\maketitle

\section{Вступ}
\quad Для реалізації структури графу я обрав представлення у вигляді матриці суміжності. Я обрав цей варіант тому, що реалізація структури та базових операцій є досить простою. Якщо ж є потреба використовувати список суміжності, то є можливість конвертувати матрицю суміжності в список суміжності та навпаки.

\section{Опис складності роботи основних операцій}
\quad Нехай граф має \(N\) вершин.

\begin{itemize}
    \item containsEdge(): \(\mathcal{O}(1)\)
    \item addVertex(): \(\mathcal{O}(N^2)\)
    \item removeVertex(): \(\mathcal{O}(N^2)\)
    \item getAdjList(): \(\mathcal{O}(N^2)\)
    \item removeEdge(): \(\mathcal{O}(1)\)
    \item addEdge(): \(\mathcal{O}(1)\)
\end{itemize}

\section{Порівняння алгоритмів топологічного сортування}
\quad У цьому пункті будемо порівнювати алгоритми топологічного сортування, що використовує DFSRecursive та алгоритм Демукрона. Одиницею вимірювання буде середній час роботи алгоритму на певному графі. Для кожного \(N\) наведеного в таблиці алгоритм буде запущений 100 разів (для різних графів), для різних констант \(p\) (від 0.1 до 1.5 з кроком 0.1), і в таблицю буде записано середнє значення.

\begin{table}
\centering
\begin{tabular}{|c|c|c|}
\hline
Алгоритм & Кількість вершин графу & Час виконання (середній) \\
\hline
DFSEnumeration & 5 & 7550 nanosec \\
& 10 & 18339 сек \\
& 100 & 1.5 сек \\
& 500 & 1.5 сек \\
& 1000 & 1.5 сек \\
\hline
Демукрона & 5 & 13201 nanosec \\
& 10 & 37073 сек \\
& 100 & 1.5 сек \\
& 500 & 1.5 сек \\
& 1000 & 1.5 сек \\
\hline
\end{tabular}
\caption{Порівняння алгоритмів топологічного сортування}
\label{tab:comparison}
\end{table}

\bibliographystyle{alpha}

\end{document}
