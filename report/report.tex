\documentclass{article}
\usepackage[english,ukrainian]{babel}
\usepackage[letterpaper,top=2cm,bottom=2cm,left=3cm,right=3cm,marginparwidth=1.75cm]{geometry}
\usepackage{amsmath}
\usepackage{graphicx}
\usepackage[colorlinks=true, allcolors=blue]{hyperref}

\begin{document}

\newpage
\section{Вступ}
\quad Для реалізації структури графу я обрав представлення графу у вигляді матриці суміжності. Цей варіант реалізації структури та базових операцій є досить простим. Якщо ж є потреба використовувати список суміжності, то є можливість конвертувати матрицю суміжності в список суміжності та навпаки.

\section{Опис складності роботи основних операцій}
\quad Нехай граф має \(N\) вершин.

\begin{itemize}
    \item containsEdge(): \(\mathcal{O}(1)\)
    \item addVertex(): \(\mathcal{O}(N^2)\)
    \item removeVertex(): \(\mathcal{O}(N^2)\)
    \item getAdjList(): \(\mathcal{O}(N^2)\)
    \item removeEdge(): \(\mathcal{O}(1)\)
    \item addEdge(): \(\mathcal{O}(1)\)
\end{itemize}

\section{Порівняння алгоритмів топологічного сортування}
\quad У цьому пункті будемо порівнювати два алгоритми топологічного сортування: алгоритм 
\\DFSEnumeration та  алгоритм Демукрона. Одиницею вимірювання буде середній час роботи алгоритму на певному розмірі графу. Для кожного \(N\) наведеного в таблиці, алгоритм буде запущений 100 разів, для кожної з констант \(p\) (від 0.1 до 1.5 з кроком 0.1), тобто в сумі 1500 запусків кожного алгоритму для кожного \(N\). Для кожного \(N\) у відповідну комірку таблиці буде записане середнє значення. 

\begin{table}[h]
\centering
\begin{tabular}{|c|c|c|}
\hline
\(N\) & DFSEnumeration & Demukron  \\
\hline
5 & 6663 nanosec & 11417 nanosec \\
10 & 15215 nanosec & 27664 nanosec \\
20 & 40654 nanosec & 76232 nanosec \\
30 & 0.12 millisec & 0.21 millisec \\
60 & 0.34 millisec & 0.56 millisec \\
80 & 0.54 millisec & 0.83 millisec \\
100 & 0.98 millisec & 1.48 millisec \\
200 & 2.48 millisec & 3.81 millisec \\
400 & 4.88 millisec & 6.68 millisec \\
600 & 10.47 millisec & 14.98 millisec \\
800 & 18.99 millisec & 26.33 millisec \\
1000 & 29.52 millisec & 41.55 millisec \\
1500 & 67.43 millisec & 96.51 millisec \\
2000 & 0.11 millisec & 0.17 millisec \\
2500 & 0.18 millisec & 0.28 millisec \\
3000 & 0.25 millisec & 0.41 millisec \\
4000 & 0.53 sec & 0.89 millisec \\
5000 & 0.82 sec & 1.25 sec \\
6000 & 1.27 sec & 1.93 sec \\
7000 & 1.72 sec & 2.67 sec \\

\hline
\end{tabular}
\caption{Порівняння алгоритмів топологічного сортування}
\label{tab:comparison}
\end{table}


\bibliographystyle{alpha}

\end{document}
